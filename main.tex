
    \documentclass{article}
\setlength{\parskip}{5pt} % esp. entre parrafos
\setlength{\parindent}{0pt} % esp. al inicio de un parrafo
\usepackage{amsmath} % mates
\usepackage[sort&compress,numbers]{natbib} % referencias
\usepackage{url} % que las URLs se vean lindos
\usepackage[top=25mm,left=20mm,right=20mm,bottom=25mm]{geometry} % margenes
\usepackage{hyperref} % ligas de URLs
\usepackage{graphicx} % poner figuras
\usepackage[spanish]{babel} % otros idiomas
\usepackage[utf8]{inputenc}
\author{Ana Lucio,
Ana Carbajal,
Eduardo Rivera,
César Luna,
René Zamora} % author
\title{Tarea 1} % titulo
\date{\today}

\begin{document} % inicia contenido

\maketitle % cabecera

\begin{abstract} % resumen
En este trabajo se hizo una introducción al curso de biomecánica, viendo conceptos básicos de mecánica, fisiología y anatomía para poder comprender los primeros temas de la materia.
\end{abstract}

\section{Introducción}\label{intro} % seccion y etiqueta
La biomecánica es una ciencia de la rama de la bioingeniería y de la ingeniería biomédica, encargada del estudio, análisis y descripción del movimiento del cuerpo, además de examinar las fuerzas en función de la estructura biológica y los efectos producidos por esas fuerzas;  se usa para el estudio de las patologías del movimiento para tratarlas o mitigar su efecto. Cada día la biomecánica mejora su sistema de diagnóstico y evaluación para obtener datos objetivos, sus patologías y el resultado de sus tratamientos.

Su objetivo es solucionar los problemas anatómicos y de movimiento que surgen de diversas condiciones a las que está sometido el cuerpo en las diversas actividades de la vida

En pocas palabras la biomecánica es un área de conocimiento que se interesa por el movimiento, equilibrio, la física, la resistencia, los mecanismos y las lesiones que pueden producirse en el cuerpo humano como consecuencia de diversas acciones físicas.

\section{Desarrollo}


La biomecánica se ayuda de otras ciencias como la mecánica y la ingeniería para que, con los conocimientos de anatomía y fisiología del cuerpo humano, poder observar, estudiar y describir el movimiento humano. 

\subsection{\textbf{*Mecánica}}
La biomecánica puede definirse como el conjunto de conocimientos interdisciplinares generados a partir de utilizar, con el apoyo de otras ciencias biomédicas, los conocimientos de la mecánica y distintas tecnologías, en primero, el estudio del comportamiento de los sistemas biológicos y, en particular, del cuerpo humano, y segundo, en resolver los problemas que le provocan las distintas condiciones a las que puede verse sometido.
En esta definición han de subrayarse algunas ideas: 

\begin{enumerate}
\item  Que a la biomecánica le compete el estudio de todos los fenómenos biológicos y, por una evidente e interesada cuestión de antropocentrismo, del cuerpo humano en especial.
\item  Que la mecánica, con un amplio apoyo tecnológico, posee métodos propios que pueden aplicarse al estudio de los seres vivos. 
\item  Que la biomecánica se ha desarrollado porque aporta un enfoque útil en el estudio y solución de los problemas que afectan al hombre -de lo contrario, probablemente, no estaríamos ocupándose de ella con tanto interés.
\end{enumerate}
El conocimiento de las propiedades mecánicas de los materiales, transferibles a los seres vivos, ha permitido entender las adaptaciones de diferentes tejidos humanos. En particular, se someten a tensiones a través de las fuerzas internas o externas a las que se someten. Estas fuerzas comportan, según su dirección, variaciones de longitud o de angulación: la deformación. La cantidad de deformación es proporcional, entre otras cosas, a la cantidad de fuerza y a las propiedades de los materiales o los tejidos. Puede ser de tipo elástico, que corresponde a una zona donde el tejido recupera su longitud inicial cuando se elimina la fuerza, o de tipo plástico, que es el caso de la zona donde el tejido se somete a cambios irreversibles. Por último, los conceptos de cinemática y cinética, aplicables al ser humano, también permiten explicar y evaluar las velocidades de movimiento, ya sea del cuerpo con respecto a su entorno o de uno de sus segmentos en relación con el resto del cuerpo o en el espacio, y sus aceleraciones. El cálculo de estas velocidades y aceleraciones es posible a partir de ecuaciones adaptadas, tanto para los desplazamientos lineales como angulares.Con la intención de divulgar los conceptos básicos de Ia mecánica que se utilizan en biomecánica ha sido elaborada la información que seguidamente ofrecemos. 


\textbf{Conceptos importantes para la biomecánica.}

\begin{itemize}
\item \textbf{Magnitudes escalares y vectoriales }
\end{itemize}

Se denomina magnitud a cualquier ente físico que se puede medir. Para representar la cantidad de determinada magnitud se emplean las unidades.
Las magnitudes se pueden clasificar en 2 tipos: escalares y vectoriales. Las magnitudes escalares son aquellas que quedan perfectamente determinadas por su valor numérico. En cambio, las magnitudes vectoriales para que resulten definidas precisan, además de su valor numérico, también su dirección, sentido y punto de aplicación.

\textbf{
\begin{itemize}
\item Sistema de coordenadas 
\end{itemize}
}

Principalmente son 3 los sistemas de coordenadas utilizados: el cartesiano, el cilíndrico y el esférico. El sistema de coordenadas cartesiano, que es el más utilizado, está formado por 3 ejes perpendiculares entre sí que se cortan en un punto común que se denomina origen de coordenadas. El sistema de coordenadas cilíndrico se utiliza para cuerpos con simetría cilíndrica y sus tres coordenadas son un ángulo y dos distancias. El sistema de referencia esférico se utiliza para localizar puntos en cuerpos con simetría esférica y las 3 coordenadas son 2 ángulos y una distancia.

\textbf{
\begin{itemize}
\item Concepto de fuerza
\end{itemize}
}
El concepto de fuerza corresponde a la acción mutua que se ejerce entre 2 cuerpos (fuerzas externas) o entre dos partes de un mismo cuerpo (fuerzas internas).
 
\textbf{
\begin{itemize}
\item Estática 
\end{itemize}
}
El equilibrio debe definirse como el estado que se poseen los cuerpos cuando la resultante de las fuerzas y los momentos actuantes son 0.
           
\[F=0 
       M=0
\]


La estática es aparte de la física que estudia los cuerpos en equilibrio estático como resultado de las fuerzas que actúan sobre él.





\textbf{
\begin{itemize}
\item Ciencia de los materiales
\end{itemize}
}
La ciencia de los materiales estudia los valores de las fuerzas internas (tensiones) y las deformaciones que experimenta un cuerpo al ser sometido a cargas mecánicas (fuerzas y momentos).

\textbf{
\begin{itemize}
\item Dinámica 
\end{itemize}
}
La mayor parte de las actividades fisiológicas conllevan el movimiento de alguna parte del cuerpo, y para su estudio y caracterización es necesario acudir a la parte de la mecánica que llamamos dinámica.\cite{1}

\subsection{*Fisiología}
El área de biomecánica y fisiología cuantitativa buscan desarrollar el estudio de las estructuras y funciones de los sistemas biológicos utilizando los principios y métodos de la mecánica, pero también implementando una perspectiva ingenieril basada en modelos matemáticos. Así, los estudios impulsados por el IIBM se han basado en la comprensión de sistemas a través de la integración de técnicas experimentales de alto rendimiento, procesamiento de datos y modelos computacionales.

Uno de los avances en esta materia es la creación de una plataforma computacional para medir la deformación local de los pulmones inducidos por ventilación mecánica. Otro es la cuantificación del flujo cardiovascular a través de imágenes de resonancia magnética, con el fin de explicar los efectos en enfermedades como la esclerosis múltiple, rupturas de aneurismas o trombosis. Pero además, se ha desarrollado el estudio del comportamiento eléctrico del corazón, proponiendo modelos matemáticos que permitan realizar una representación de éste.

Asimismo, grupos de investigación del instituto han observado las respuestas de las neuronas ante lesiones en la médula espinal, por medio del estudio de modelos de sistemas neuronales de insectos, que se asemejan al funcionamiento de la médula humana. Ello hace posible generar un prototipo de red de neuronas por medio de micro neurotransmisores y así, aplicar las interpretaciones a reacciones del sistema nervioso del ser humano. \cite{3}

\subsection{*Anatomía}
Se denomina biomecánica al análisis de la mecánica del movimiento del cuerpo humano. Se trata de la ciencia que explica cómo y por qué el cuerpo humano se mueve de la forma que lo hace. Esto incluye la interacción existente entre la persona que ejecuta el movimiento y el equipamiento o el entorno.

En biomecánica, se considera que cualquier movimiento parte de una posición anatómica. Una posición anatómica es aquella en la que una persona está situada de pie, con la vista hacia delante, los brazos a los laterales del cuerpo con las palmas hacia el frente, con los pies ligeramente separados en la zona de los talones y los dedos de los pies señalando hacia delante. Existen tres planos anatómicos o cardinales en la posición anatómica, según se describe a continuación.

El plano sagital o mediano divide el cuerpo en dos lados (derecho e izquierdo), con algunas excepciones: los movimientos de flexión (reducción del ángulo de una articulación o doblar la articulación) y de extensión (aumentar el ángulo de la articulación o extender la articulación) se producen en el plano sagital.

La segunda división del cuerpo se realiza a través del plano frontal o coronal, que distingue la parte delantera y la parte trasera del cuerpo. Una vez más, hay algunas excepciones: los movimientos de abducción (separar una extremidad de la línea central del cuerpo) y de aducción (acercar una extremidad a la línea central del cuerpo) se producen en el plano frontal.

Por último, el plano transversal u horizontal divide el cuerpo en una parte superior y una parte inferior. Los movimientos de rotación se producen en el plano transversal. Los patrones diagonales de movimiento se producen cuando los componentes de estos tres planos cardinales de movimiento se combinan al mismo tiempo. \cite{2}


\subsection{*Ingeniería en la biomecánica}

La biomecánica es una disciplina de la ingeniería biomédica que emplea los principios de la mecánica para estudiar el efecto de la energía y de las fuerzas de la materia en sistemas vivos para modelarlos. 
Forma parte de la historia científica y ha influenciado la investigación de matemáticos, ingenieros, físicos, biólogos y médicos. Se considera que su progreso es resultado de la integración y aplicación de las matemáticas, los principios físicos, la fisiología y metodologías de ingeniería, los avances en los métodos experimentales y de la instrumentación para entender y resolver problemas de Ingeniería Biomédica. 

La biomecánica genera aplicaciones importantes, que son esenciales para el mejoramiento de la existencia humana, entre las que se puede citar el desarrollo de modelos de los sistemas del cuerpo humano, como el sistema músculo-esquelético, el respiratorio, el cardiovascular y el cardiopulmonar. Desde el punto de vista tecnológico se puede indicar que con base a esta disciplina se desarrollan dispositivos para asistir a tareas para mejorar el rendimiento deportivo, evaluar condiciones de trabajo, para rehabilitación física, ejecución de cirugía ortopédica, diseño de prótesis y órtesis. 

En el área del deporte es posible realizar análisis de movimiento con laboratorios de análisis del comportamiento en el deporte, de aprendizaje y control motor, de biomecánica y salud, de análisis y optimización del entrenamiento (Centro de Investigación del Deporte, 2008), que son laboratorios que incorporan sistemas de electromiografía, dinamometría, acelerometría, goniometría y fundamentalmente se tienen laboratorios de marcha que permiten obtener información cuantitativa y cualitativa del entrenamiento y del rendimiento físico.\cite{4}


\section{Conclusiones}
Podemos concluir que la biomecánica es una rama de la bioingeniería y de la ingeniería biomédica demasiado amplia, ya que esta abarca demasiados campos y ramas, esto es lo que la hace interesante, ya que no solo es la parte médica como lo es la anatomía y fisiología, sino también es la parte de ingeniería que es lo cinemático, dinámico, y mecánico. La biomecánica puede ser aplicada en demasiadas cosas ya que esta se basa en el movimiento del cuerpo humano, explicando el porqué el cuerpo humano hace ciertos movimientos y también el porqué no los hace.


\bibliography{bib}
\bibliographystyle{plainnat}

\end{document}