  \documentclass{article}
\setlength{\parskip}{5pt} % esp. entre parrafos
\setlength{\parindent}{0pt} % esp. al inicio de un parrafo
\usepackage{amsmath} % mates
\usepackage[sort&compress,numbers]{natbib} % referencias
\usepackage{url} % que las URLs se vean lindos
\usepackage[top=25mm,left=20mm,right=20mm,bottom=25mm]{geometry} % margenes
\usepackage{hyperref} % ligas de URLs
\usepackage{graphicx} % poner figuras
\usepackage[spanish]{babel} % otros idiomas
\usepackage[utf8]{inputenc}
\usepackage{float}
\author{Ana Lucio,
Ana Carbajal,
Eduardo Rivera,
César Luna,
René Zamora} % author
\title{Tarea 5} % titulo
\date{\today}

\begin{document} % inicia contenido

\maketitle % cabecera


\section{Biomecánica de la mano}
La mano tiene múltiples funciones, una de las más importantes es la de tocar (función sensitiva) y la de prensión (función motora). La mano puede conseguir esto con sutileza y precisión; su posición funcional es aquella donde la muñeca se encuentra en extensión ligera e inclinación cubital leve, los dedos ligeramente flexionados, y el pulgar en semiposición, a 40° de antipulsión y a 20° de abducción.

Existen diferentes tipos de prensión, entre los cuales están:

\begin{itemize}
\item La prensión terminal de los dedos.
\item La prensión subterminal de los dedos.
\item La prensión subtérmino lateral de los dedos.
\item La prensión digitopalmar completa.
\item La prensión digitopalmar incompleta.
\item La prensión laterolateral de los dedos.
\end{itemize}

\textbf{Ejes de los dedos}
Cuando la mano está en su posición natural los dedos están separados entre sí y el eje de los dedos pasa por el edo medio, esté dedo sirve de referencia, a partir de la cual se produce la aproximación o separación de los demás dedos; así mismo, existe un paralelismo entre los ejes de los tres últimos dedos y una divergencia entre los tres primeros.

\textbf{Articulaciones metacarpofalángicas}

Estas articulaciones son de tipo condíleo, permiten movimientos activos de flexotensión, palmar y dorsal, abducción y aducción, y pequeños movimientos pasivos de rotación axial.
La flexión activa alcanza casi 90° en el dedo índice y va aumentando de manera progresiva hasta el dedo meñique cuando se flexionan todos los dedos a la vez; la flexión aislada de un dedo está limitada por el ligamento palmar interdigital.
El dedo índice posee la mayor amplitud de movimientos de abducción y aducción pueden llegar hasta 30°, los movimientos son realizados independientemente de los demás dedos. Así mismo, la rotación axial interna del índice puede llegar hasta 45°, mientras que su rotación axial externa es prácticamente nula.

\textbf{Articulaciones interfalángicas}

Estas articulaciones son de tipo troclear, permiten sólo un tipo de movimiento, siendo este el de flexoextensión. La flexión activa de las articulaciones interfalángicas proximales sobrepasa los 90°, aumentando progresivamente desde el segundo hasta el último dedo, llegando a 135° en el meñique.

La extensión activa de las articulaciones interfalángicas es nula; en las articulaciones distales puede haber un mínimo movimiento de 5° (sometido a variaciones individuales). La extensión pasiva es nula en las articulaciones interfalángicas proximales, aunque pueden llegar a 30° en las articulaciones interfalángicas distales.

Los movimientos de lateralidad pasivos pueden alcanzar 5° en las articulaciones interfalángicas distales, pero son nulos en las proximales, aquí es donde la estabilidad lateral condiciona la potencia de prensión de la mano.

\textbf{Tendones de los músculos flexores de los dedos}

Estos músculos se originan en la epitróclea humeral y se dirigen hacia la cara palmar. El flexor común superficial de los dedos es flexor de la segunda falange debido a su inserción en las caras laterales de esta y por tanto no tiene acción sobre la tercera falange; mientras que el flexor común profundo de los dedos se inserta en la base de la tercera falange, convirtiéndolo en el único encargado de la flexión de la tercera falange, en los movimientos normales de la mano, la flexión de la tercera falange obliga a la flexión de la segunda, esto debido a que no existe un extensor selectivo que sea capaz de sostenerla.

\textbf{Tendones de los músculos extensores de los dedos}

Estos músculos nacen en el epicóndilo humeral y se dirigen hacia la cara dorsal, son músculos extrínsecos que transcurren por correderas a nivel de la muñeca y por debajo del ligamento anular posterior del carpo. El exterior común de los dedos es el extensor de la primera falange sobre el metacarpiano.
El extensor propio del índice y del meñique están unidos al extensor común de los dedos, permitiendo la extensión aislada del índice y del meñique con los demás dedos en flexión.

\textbf{Acción de los músculos interóseos y lumbricales}

Los músculos interóseos y lumbricales son imprescindibles para realizar los movimientos de lateralidad y de flexoextensión de los dedos. Los movimientos de lateralidad dependen de la dirección del cuerpo muscular, de forma que cuando se dirige al eje de la mano son los responsables de la separación de los dedos. Su acción sobre la flexoextensión de los dedos es la más importante debido a que su complejidad depende la función principal de la mano que es la presión. 

\textbf{Acción del extensor común }

El extensor común de los deseos sólo actúa cuando la muñeca y las articulaciones metacarpofalángicas están en flexión.
Acción de los músculos interóseos 
Los músculos interóseos son flexores de la primera falange y extensores de la segunda y tercera. Cuando la articulación metacarpofalángica está en extensión, la cubierta dorsal de los interóseos se sitúa en el dorso del cuello del primer metacarpiano y así extender la segunda y la tercera falanges. 
Cuando la articulación metacarpofalángica se flexiona, la cubierta dorsal de los interóseos se desliza sobre el dorso de la base de la primera falange, lo que desemboca en  la flexión de al primera falange, pero las expansiones laterales quedan relajadas por la cubierta dorsal, y pierden por tanto su acción extensora sobre la segund ay la tercera falanges.

\textbf{Acción de los músculos lumbricales }

Estos músculos se encuentran en un plano más palmar que el ligamento transverso intermetacarpiano, posee un ángulo de incidencia de 35° con respecto a la primera falange, y además, su inserción distal tiene lugar en un plano distal lo que permite la extensión de la segunda y tercera falanges.

\textbf{Ligamento retinacular}

Se encuentra en cada lado de la articulación interfalángica, sin conexión muscular alguna, Se introduce en la cara palmar de la primera falange y se dirige a las cintillas laterales del extensor común en el dorso de la segunda falange. De esta forma, la extensión de la articulación interfalángica proximal tensa el ligamento retinacular y provoca de manera automática la extensión de la articulación interfalángica distal en la mitad de su recorrido.

\textbf{Músculos de la eminencia hipotenar} 

En la zona de los músculos de la eminencia hipotenar se encuentran tres músculos que actúan directamente sobre el dedo mequiñe. Estos son: el oponente, el flexor corto y el aductor.

El oponente actúa sobre el quinto metacarpiano imprimiendo un movimiento de flexión y rotación alrededor de su eje longitudinal de manera que su porción anterior se dirige hacia fuera en dirección al dedo pulgar.

El flexor corto flexiona la primera falange sobre el primer metacarpiano, al tiempo que separa al dedo meñique del eje de la mano.
\newpage
El aductor y flexor corto son abductores del dedo meñique con respecto al eje de la mano, además, son flexores de la primera falange y extensores de la segunda y tercera en una acción semejante a la de los interóseos dorsales.

\textbf{El pulgar }

La articulación trapeciometacarpiana es básica dentro de la biomecánica del pulgar, está formada por la carilla articular inferior del trapecio, clásicamente definida como “en silla de montar” se articula con la extremidad proximal del primer metacarpiano. 

El pulgar realiza movimientos por la articulación trapeciometacarpiana, estos son: de antepulsión y retropulsión y aducción y abducción. En los primeros dos, también denominados de anteposición y retroposición, el primer metacarpiano se dirige hacia delante o hacia atrás, y el pulgar se sitúa por encima de la palma de la mano en la anteposición y a nivel plano de la palma en retroposición. En los dos últimos, cuando se realiza la aducción se dirige hacia abajo acercándose el pulgar hacia la mano mientras que en la abducción el primer metacarpiano se dirige hacia arriba. 

En el pulgar actúan dos tipos de articulaciones, que son; la metacarpofalángica, que es una articulación de tipo condíleo y permite en teoría dos tipos de movimientos, movimiento axial activo y pasivo; y la articulación interfalángica, que es de tipo troclear y permite solo movimientos de flexo extensión, la flexión no alcanza más que 75-80°.

En el pulgar actúan de igual forma dos músculos, el extrínseco y el intrínseco. El primero actúa de tal forma que el extensor corto del pulgar realiza la extensión de la primera falange, esto convierte al extensor como el verdadero abductor del pulgar; el extensor largo del pulgar lleva al primer metacarpiano hacia dentro y hacia atrás por lo que es también aductor y extensor del primer metacarpiano; por último, el flexor largo desempeña un papel insustituible en el movimiento del flexor de la tercera falange.

La mano tiene una función específica, esta es simular un efecto de pinza o presión, debido a la oposición del pulgar a los demás dedos. Esta oposición resulta de la coordinación de varios movimientos como son la antepulsión y aducción del primer metacarpiano, junto con la rotación axial del primer metacarpiano  y de la primera falange. El pulgar es el más importante de los dedos de la mano gracias a su movilidad, fuerza y a su capacidad irremplazable de oponerse a cada uno de los demás dedos y a la palma de la mano. 

\textbf{La mano} 

Esta extremidad del cuerpo humano cumple diferentes funciones que son fundamentales en nuestro día a día; una de ellas es la de prensión, que es una función motora; y la otra es que nos permite identificar objetos por medio del tacto, función sensitiva.

Como bien sabemos, existen infinidad de combinaciones cuando hablamos de elementos móviles de la mano para tomar objetos y adaptarse a su forma. Esto dependerá de la fuerza que se aplique al momento de realizar el acto de prensar. Existen varios tipos: la prensión terminal de los dedos que se realiza por la oposición de pulpejo del dedo pulgar con la punta de los demás dedos, sobre todo el dedo índice o en su defecto el dedo de en medio; la prensión subterminal que se lleva a cabo por la oposición del pulgar con cualquiera de los demás dedos, este tipo es más aplicable en objetos de mediano tamaño: la prensión subtérmino lateral que es fuerte y eficaz, esta se realiza entre la cara palmar del pulpejo del dedo pulgar y la cara lateral radial del dedo índice; la prensión digitoplamar completa que se utiliza de manera habitual para sostener una llave cuando estamos intentando abrir una puerta; la prensión digitopalmar incompleta, que es en la que participan todos los dedos en oposición a la palma de la mano, excepto el dedo pulgar, es también una modalidad de fuerza, pero no es tan sólida como la anterior ya que los objetos más pesados pueden escapar en la dirección de la muñeca; y por último, la prension laterolateral de los dedos, esta es una modalida accesoria que en general se realiza entre el dedo indice y el medio para sostener objetos pequeños y livianos, como por ejemplo un cigarrillo o juguetear con un lápiz.

La disposición anatómica de la mano permite entender su gran versatilidad en la manipulación de objetos y ajustes posicionales de acuerdo a las necesidades en la ejecución de patrones funcionales. Correlacionar sus unidades arquitectónicas con el complejo biomecánico de cada una de ellas, permite entender que la función prensil de la mano depende de la integridad de la cadena cinética de huesos y articulaciones extendida desde la muñeca hasta las falanges distales. 

\cite{voegeli2000lecciones}
\newpage

\section{Conclusiones}
En esta actividad se vio lo que es la biomecánica de la mano, una parte esencial del cuerpo humano con la cual realizamos la mayoría de nuestras actividades, vimos cada una de las partes que la conforman y el movimiento de cada una de ellas, es interesante aprender cómo una parte que usamos y vemos a diario sea compuesta de tantas articulaciones, huesos, músculos, acciones, etc., que si te dedicas a estudiarlo a detalle se vuelve algo mucho más complejo de lo que puedes observar a simple vista.

Este ensayo nos será de mucha ayuda para el proyecto que estamos realizando ya que podemos entender más a detalle cómo está compuesta una mano, hablando específicamente de movimientos ya que es una cuestión muy compleja la cual es fundamental al momento de fabricar prótesis.



\bibliography{bib}
\bibliographystyle{plainnat}
\end{document}